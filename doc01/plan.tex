\documentclass[a4paper,12pt]{article}

\usepackage{titlesec} %za renewcommand
\usepackage{titling} %za renewcommand
\usepackage{ragged2e} %za na pr. flushleft
\usepackage{fancyhdr}
\usepackage{enumitem}
\usepackage{pdflscape}
\usepackage{array}
%\usepackage{tabularx}
\usepackage{makecell} %multiline cells + other non multiline cell are centered horiz. and vert.
\usepackage[table]{xcolor}
\usepackage{geometry}
\usepackage{hyperref}
\usepackage{booktabs}
\usepackage{color, soul}

\geometry{left=1.5cm,right=1.5cm,top=2cm,bottom=2cm}

\setcounter{page}{0}
\setcounter{section}{-1}

\pagestyle{fancy}
\fancyhf{}
\lfoot{Powerd by \LaTeX}
\cfoot{Verzija 1.0}
\rfoot{Stran: \thepage}

\newcommand{\mlc}[1]{\raisebox{0ex}{\makecell{#1}}}

\renewcommand{\headrulewidth}{0.5pt}
\renewcommand{\footrulewidth}{0.5pt}

\renewcommand{\maketitle}{
	\begin{center}
	{\huge\bfseries
	\thetitle}
	\end{center}

	\vspace{1cm}

	\centerline{\textbf{Naročnik: Šaj d.o.o.}}
	\centerline{\textbf{Vodja projekta: Anton Zhezhov}}

	\vspace{1cm}

	\leftline{\textbf{Začetek: 09.10.2019 } \hfill \textbf{Konec: 31.01.2020}}

	\vspace{4cm}

	\begin{center}
	
			\begin{tabular}{c|c|c|c}
					Ime in Priimek&Vloga	  &e - Naslov						&Opomba\\
				\hline
					Anton Zhezhov &Preverjanje&anton.zhezhov@student.um.si & 		\\
				\hline
					Žiga Zorc	  &Razvoj	  &ziga.zorc@student.um.si 	   &		\\
			\end{tabular}
	
	\end{center}
}

\let\oldtitleline\titleline
\renewcommand{\titleline}{\oldtitleline*}

\title{Projekt CSUPP}
\author{Anton Zhezhov}


\begin{document}

%underline color options for "paramteri"
\setul{0.6ex}{0.17ex}
\definecolor{Red}{rgb}{1,0.0,0.6}
\setulcolor{Red}



\maketitle

\setlength{\titlewidth}{16cm}

\newpage

\section{Naročnikove zahteve}
	\subsection{Splošne informacije}
			\begin{center}
				\begin{tabular}{c|c}
						Dokument & Verzija 1.0 \\
						\hline
						Naročnik & Šaj d.o.o. \\
						\hline
						Lokacija dokumenta & link \\
						\hline
						Odgovorna oseba & Direktor podetja Šaj d.o.o. \\
				\end{tabular}
			\end{center}
	\subsection{Zahteve}
 
	\hspace{1em} V podjetju Šaj d.o.o. se ukvarjamo z razvojem inovativnih rešitev na področju 
	avtomatizacije in digitalizacije upravljanja poslovnih prostorov. Pri načrtovanju 
	naših rešitev dajemo velik poudarek na okoljsko trajnost, energetsko učinkovitost 
	ter ergonomičnost produktov, saj se zavedamo, da omenjene lastnosti pozitivno vplivajo 
	tako na izboljšano uporabniško izkušnjo kot na optimizacijo poslovanja skozi nižanje 
	stroškov.
	
	V podjetju smo prepoznali pomanjkanje rešitev, ki bi celovito naslovile 
	problem zastarelosti poslovnih prostorov. V ta namen načrtujemo razvoj centralnega 
	sistema za upravljanje poslovnega prostora, s čimer se nadejamo preboja na trg 
	in s tem izboljšanja poslovnega uspeha. Projekt že ima izoblikovano idejno zasnovo, 
	in sicer tako glede strojne opreme kot izgleda in funkcionalnosti. Sedaj smo v fazi 
	iskanja resnega partnerja, ki bi prevzel razvoj programske opreme. Ker želimo preveriti 
	osnovni koncept in delovanje centralnega sistema za upravljanje prostora, naj bo program 
	napisan v obliki simulatorja. Najprej potrebujemo preprost simulator brez grafičnega 
	vmesnika, ki bo izdelan kot konzolna aplikacija v jeziku C++ v integriranem razvojnem 
	okolju Visual Studio. Od simulatorja pričakujemo brezhibno in robustno delovanje v 
	operacijskem sistemu Windows. Poleg tega mora biti simulator hiter in preprost za uporabo. 
	Simulator naj omogoča krmiljenje temperature, vlage in osvetljenosti prostora. Predpogoj 
	je, da uporabnik v tekstovno datoteko vpiše želene ambientalne lastnosti v obliki: 
	
	TEMPERATURA: vrednost 

	VLAZNOST: vrednost v obliki relativne vlažnosti [\%] 

	OSVETLJENOST: vrednost v luksih [lx] 
	\\
	V datoteki naj bo še: 

	INTERVAL TEMPERATURE: [10,40]
	
	STOPNJA VLAZNOSTI: [30,60]

	INTERVAL OSVETLJENOSTI: [10,10000] 
	\\
	Simulator naj pred pričetkom prebere 
	vrednosti iz datoteke, nato pa naj omogoča izbiro med tremi načini delovanja: 

	1. Testni način: Uporabnik v program vnese dejansko temperaturo v prostoru. Računalnik vneseno 
	temperaturo pretvori v ostale relevantne merske enote. Nato naj izračuna razliko do 
	želene temperature (v vseh izbranih merskih enotah) in izvede ukaz za regulacijo 
	temperature. Analogno naj simulator omogoča vpis, izračun in izvedbo ukazov še za 
	vlažnost in osvetljenost. Simulacija se izvaja, dokler je ne prekine uporabnik. 
	
	2. Avtomatski način: Računalnik naj si izmisli dejansko temperaturo na intervalu 
	podanem v datoteki, pri čemer jo pretvori v najpomembnejše preostale merske enote. 
	Izmisli naj si še relativno stopnjo vlažnosti, in sicer med 30 in 60 \%, ter osvetljenost 
	na intervalu z datoteke. Nato naj za vsako posamezno meritev izračuna odstopanje od 
	želenih vrednosti ter izvede ukaze za popravek. Simulator naj izvede 100 meritev, pri 
	čemer izvede posamezno meritev vsake 3 sekunde. Na koncu simulacije naj izračuna 
	povprečno vrednost meritev ter povprečno odstopanje od želenih vrednosti za posamezne 
	\hyperlink{subsection.1.8}{\ul{parametre}}. 

	3. Avtomatski način 2: Simulator naredi isto kot v točki 2, pri čemer 
	naj uporabniku omogoča izbiro pri številu meritev in časovnem razmiku med njimi. 
	Izvajalec mora natančno slediti vsem internim standardom in poskrbeti za dokumentacijo. 
	
	Sestavni del projekta sta tudi razvijalska dokumentacija in uporabniški priročnik. 
	Od izvajalca pričakujemo, da do 24. 10. 2019 do 23.55 odda plan projekta, ki vključuje ceno. 
	Program in dokumentacija morata biti oddana najkasneje 23. 1. 2020 do 23.55. 
	Projekt bo plačan po posameznih zaključenih fazah. Za vsak teden zamude bo odbitih 10 \% plačila. 
	\\
	
	\leftline{Maribor 01.10.2019} 

	\hfill Direktor podjetja Šaj d.o.o.	

\newpage

\section{Plan projekta}

	\subsection{Tabela}

	\subsection{Kratek opis problema}

		\hspace{1em} Podjetje Šaj d.o.o. (v nadaljevanju naročnik) je dne 1. 10. 2019 naročilo 
		razvoj centralnega sistema za upravljanje poslovnega prostora.
		
		Naročnik želi optimizirati svoje poslovne prostore z avtomatiziranim sistemom, 
		ki meri in upravlja s paramteri. Sistem je preprost "simulator", ki je sposoben 
		prilagajanja \hyperlink{subsection.1.8}{\ul{parametrov}} tako avtomatsko kot na specifične uporabnikove zahteve.
		\subsubsection{Globalni cilji(globalne zahteve), ki jih želimo s produktom doseči}

		\begin{itemize}
				\item Izdelati simulator, ki primerno regulira {\hyperlink{subsection.1.8}{\ul{parametre}}} v prostoru
			\item Simulator mora biti hiter in preprost za uporabo
		\end{itemize}

		\subsubsection{Omejitve}

				\begin{itemize}
					\item Programski jezik: C++
					\item Operacijski sistem: Windows
					\item Izdelan mora biti kot simulator
					\item Konzolna aplikacija oz. brez grafičnega vmesnika
				\end{itemize}

		\subsubsection{Rok za zaključitev projekta, skupni stroški}

				\begin{itemize}
					\item Do 22.10.2019 do 23:55 oddan plan projekta 
					\item Do 23.01.2019 do 23:55 oddan projekt
				\end{itemize}

		\subsubsection{Funkcije}

				\begin{itemize}
					\item Pretvarjanje temperature v merske enote
					\item Razlika do želene temperature
					\item Regulacina temperature
					\item Računalnik simulira(ustvari svoje vrednosti) in na to	izračuna odstop od ustvarjene vrednosti
				\end{itemize}

		\subsubsection{Pomembne karakteristike}

				\begin{itemize}
					\item Preprost za uporabo oz. intuitiven in hiter
					\item Delovanje v OS Windows
				\end{itemize}

		\subsubsection{Neizvedljive zahteve}

				\begin{itemize}
					\item Brezhibnost
					\item Robustnost
				\end{itemize}

		\subsubsection{Označevanje verzij}

				\begin{itemize}
					\item Verzija: vx.y\_DDMMLLLL
					\item x - velike spremembe, y - manjše spremembe
					\item Primer: v3.1\_17112019
				\end{itemize}

	\subsection{Zagotavljanje kakovnosti (Načrt preverjanja)}

%underline color options for "prilogi"
\setul{0.6ex}{0.17ex}
\definecolor{Red}{rgb}{1,0.5,0.0}
\setulcolor{Red}



		\subsubsection{Objekti preverjanja}

			\begin{itemize}
				\item D1 	Naročnikove zahteve
				\item D2 	Plan projekta
				\item D3 	Sistemske specifikacije
				\item D4 	Testne primere
				\item D5 	Poročilo o preverjanju
				\item D6 	Načrtovalsko dokumentacijo
				\item D7 	Uporabniški priročnik	
			\end{itemize}

			Glede na izbran model razvoja obstajajo delni in končni produkti, 
			ki jih je potrebno na koncu vsake faze preveriti (glej tabelo 
			Pregled po produktih in aktivnostih). Kompleten terminski plan 
			je podan v nadaljevanju tega dokumenta. Končni produkt 
			predstavljajo dokumenti D1-D7.

			\begin{enumerate}[label=\alph*)]
				\item Preverjanje programa v1.0

					Program v1.0 bomo preverili s pregledom izvorne kode 
					(stil kodiranja, skladnost s standardom) in testiranjem. 
					Pripravljeni bodo določeni testni vzorci in postopki, 
					ki jih bo natančneje definiral dokument Testni primeri. 
					Preverjanje izvaja preverjevalec. Po preverjanju se izpolnijo 
					pisna poročila o najdenih neustreznostih. Na podlagi teh poročil 
					se izvede odpravljanje neustreznosti. Najprej se bodo preverili 
					tipični testni vzorci, če pri njih ne najdemo resne hibe, se 
					izvedejo tudi ostali testi. Ne izvaja se nobenih regresijskih testov.
				\item Preverjanje programa v2.0

					Program v2.0 bomo preverili s pregledom izvorne kode 
					(stil kodiranja, skladnost s standardom) in testiranjem. Pripravljeni 
					bodo določeni testni vzorci in postopki, ki jih bo natančneje definiral 
					dokument Testni primeri. Preverjanje izvaja preverjevalec. Po preverjanju 
					se izpolnijo pisna poročila o najdenih neustreznostih. Izvedejo se vsi 
					testi (regresijsko testiranje).	
			\end{enumerate}

			\textbf{Uporabljene bodo naslednje strategije (podroben opis v \hyperlink{subsection.1.9}{\ul{prilogi}}})

				\begin{itemize}
					\item prisotnost zahtev (Z)
					\item prepovedane vrednosti - za preverjanje robustnosti (R)
					\item mejne vrednosti (M)
					\item ugibanje napak oziroma nepravilnosti (U)
				\end{itemize}
\newpage

\begin{landscape}

	\subsection{Naloge in rezultirajoči dokumenti (izbran razvojni model)}
		\subsubsection{Pogled po produktih in aktivnostih}
		\vspace{4cm}
		\begin{center}
		\footnotesize
		\rowcolors{1}{purple!30!}{white}
		\begin{tabular}{|c|c|c|c|c|c|c|c|c|}
				  \hline
				  &Produkt&\makecell{Planirana \\ kompleksnost} &\makecell{Dejanska \\ kompleksnost}&\makecell{Odgovorna oseba \\ za produkt}&\raisebox{0ex}{\makecell{V\&V metoda}}&\raisebox{0ex}{\makecell{Odgovorna \\ oseba za V\&V}}&\makecell{Način sporočanja \\ o V\&V}&Opomba\\
				\hline
				D1&\makecell{Naročnikove zahteve}&1.5 strani& &naročnik&&&&\\
				\hline
				D2&Plan projekta&6 strani& &Anton Zhezhov&splošni pregled&Anton Zhezhov&&\\
				\hline
				D3&\makecell{Sistemske specifikacije}&10 strani&&Anton Zhezhov&splošni pregled&Anton Zhezhov&&\\
				\hline
				  &Program v1.0&700 LOC&&Žiga Zorc&sploš. pregled + test&Anton Zhezhov&&\\
				\hline
				D4&Testni primeri&\makecell{50 testnih \\ primerov}&&Anton Zhezhov&sploš. pregled + test&Anton Zhezhov&&\\
				\hline
				D5&Testno poročilo&6 strani&&Anton Zhezhov&splošni pregled&Anton Zhezhov&&\\
				\hline
				D6&\makecell{Načrtovalska \\ dokumentacija}&5 strani&&Anton Zhezhov&splošni pregled&Anton Zhezhov&&\\
				\hline
				D7&\raisebox{0ex}{\makecell{Uporabniški priročnik}}&8 strani&&Žiga Zorc&splošni pregled&Žiga Zorc&&\\ %raisebox is resizeing the cell but in this case resize is set to 0 so that te cell can be colored properly.
				\hline
				  &Program v2.0&1300 LOC&&Žiga Zorc&sploš. pregled + test&Anton Zhezhov&&\\
				\hline
				  &\raisebox{0ex}{\makecell{Kompleten \\ produkt}}&1500 LOC&&vsi&&&&\\
				\hline

		\end{tabular}
		\end{center}

\newpage

		\subsubsection{Rok in stršoki}
		\vspace{3cm}
		\begin{center}
		\rowcolors{1}{purple!30!}{}
		\begin{tabular}{|c|c|c|c|c|c|c|c|c|c|}
				\hline
				&Aktivnost&Planiran rok&Dejanski rok&Planirani napor&Planirani stroški&\raisebox{0ex}{\makecell{Dejanski \\ napor}}&\raisebox{0ex}{\makecell{Dejanski \\ stroški}}&Izvajalec&\makecell{Odgovorna \\ oseba}\\
				\hline
				A1&\makecell{Planiranje projekta \\ in analiza zahtev}&22.10.2019&&4&400&&&&\\
				\hline
			 A2&Načrtovanje&03.11.2019&&4&400&&&&\\
				\hline
			 A3&\makecell{Implementacija \\ programa v1.0}&03.01.2020&&4&400&&&&\\
				\hline
				A4&\raisebox{0ex}{\makecell{Implementacija \\ programa v2.0}}&16.01.2020&&6&600&&&&\\
				\hline
			 A5&\makecell{Načrtovanje testnih \\ primerov}&03.01.2020&&2&200&&&&\\
				\hline
			 A6&\raisebox{0ex}{\makecell{Preverjanje programa \\ v1.0}}&09.01.2020&&3&300&&&&\\
				\hline
			 A7&\makecell{Preverjanje programa \\ v2.0}&TBD&&2&200&&&&\\
				\hline
				A8&\raisebox{0ex}{\makecell{Izdelava kompletne \\ dokumentacije}}&16.01.2020&&7&700&&&&\\
				\hline
			A9&\makecell{Prevzem}&23.01.2020&&1&100&&&&\\
				\hline
			A1&\makecell{Skupaj naport - stroški}&&&&3300&&&&\\
				\hline

		\end{tabular}

				\vspace{1cm}

				Enota napora: človek-dan

				Stroški enote napora: 100 EUR
		\end{center}


\end{landscape}

\newpage

	\subsection{Resursi}

		\subsubsection{Osebje}
			\begin{center}
			\rowcolors{1}{purple!30!}{}
			\begin{tabular}{|c|c|>{\centering}m{0.43\textwidth}|c|}
				\hline
				&Oseba&Aktivnost&Vloga\\
				\hline
			  P1&Direktor podetja&
			\begin{itemize}
				\item nadzor
				\item prevzem
			\end{itemize}&naročnik\\
				\hline
			  P2&Anton Zhezhov&
				\begin{itemize}
					\item načrtovanje testnih primerov
					\item testiranje
					\item planiranje projekta
					\item izdelava načrtovalske dokumentacije
				\end{itemize}&preverjevalec\\
				\hline
			  P3&Žiga Zorc&
				\begin{itemize}
					\item analiza zahtev
					\item načrtovanje
					\item implementacija programa v1.0
					\item implementacija programa v2.0
					\item prevzem
				\end{itemize}&razvojnik\\
				\hline
			\end{tabular}
			\end{center}
		
		\subsubsection{Potrebna programska orodja, knjižnice}
			\begin{center}
			\begin{tabular}{|c|c|}
					\hline
					\rowcolor{purple!30!} Orodje& Namen, funkcija\\
					\hline
					Microsoft Visual C++& Kodiranje, odpravljanje neustreznosti\\
					\hline
					\LaTeX &Vodenje dokumentacije\\
					\hline
					TBD& Merilnik kompleksnosti\\
					\hline
			\end{tabular}
			\end{center}

		\subsubsection{Potrebna strojna oprema}
			\begin{center}
			\begin{tabular}{|c|c|}
					\hline
					\rowcolor{purple!30!} Orodje& Namen, funkcija\\
					\hline
					PC& Kodiranje, odpravljanje neustreznosti, vodenje dokumentacije, testiranje\\
					\hline
					Printer&Izpis dokumentacije\\
					\hline	
			\end{tabular}
			\end{center}


	\subsection{Razdelitev stroškov}
		\qquad \qquad \textcolor[HTML]{C50918}{\hyperlink{subsubsection.1.4.2}{Točka 1.4.2}}

\newpage

\begin{landscape}

	\subsection{Terminski plan projekta}

		\vspace{2.5cm}
		\begin{center}
		\resizebox{24cm}{!}{
				\rowcolors{1}{purple!30!}{}
		\begin{tabular}{|c|c||c|c|c|c|c|c|c|c|c|c||c|c|c|c|c|c|c|c|c|c||c|c|c|c|c|c|c|c|c|c||}
				\hline
				&Aktivnost&\multicolumn{30}{|c|}{Časovna skala}\\
				\hline
				&&1&\mlc{1\\2}&2&\mlc{2\\3}&3&\mlc{3\\4}&4&\mlc{4\\5}&5&\mlc{5\\6}&6&\mlc{6\\7}&7&\mlc{7\\8}&8&\mlc{8\\9}&9&\mlc{9\\10}&10&\mlc{10\\11}&11&\mlc{11\\12}&12&\mlc{12\\13}&13&\mlc{13\\14}&14&\mlc{14\\15}&15&\\
				\hline
				A1&\mlc{Planiranje projekta in analiza zahtev}&+&+&+&+&+&+&+& & & & & & & & & & & & & & & & & & & & & & & \\
				\hline
				A2&\mlc{Načrtovanje}& & & & & & & &+&+&+&+&+& & & & & & & & & & & & & & & & & & \\
				\hline
				A3&\mlc{Implementacija programa v1.0}& & & & & & & & & & & &+&+&+&+&+&+&+&+&+&+& & & & & & & & & \\
				\hline
				A4&\mlc{Implementacija programa v2.0}& & & & & & & & & & & & & & & & & & & & & &+&+&+&+&+& & & & \\
				\hline
				A5&\mlc{Načrtovanje testnih primerov}& & & & & & & & & & &+&+&+&+& & & & & & & & & & & & & & & & \\
				\hline
				A6&\mlc{Preverjanje programa v1.0}& & & & & & & & & & & &+&+&+&+&+&+&+&+&+&+& & & & & & & & & \\
				\hline
				A7&\mlc{Preverjanje programa v2.0}& & & & & & & & & & & & & & & & & & & & & &+&+&+&+&+& & & & \\
				\hline
				A8&\mlc{Izdelava kompletne dokumentacije}& & & & & & & & & & & & & & & & & & & & & & & & & & &+&+& & \\
				\hline
				A9&\mlc{Prevzem}& & & & & & & & & & & & & & & & & & & & & & & & & & & & &+& \\
				\hline
				&Dokument (skrajni rok)&1&\mlc{1\\2}&2&\mlc{2\\3}&3&\mlc{3\\4}&4&\mlc{4\\5}&5&\mlc{5\\6}&6&\mlc{6\\7}&7&\mlc{7\\8}&8&\mlc{8\\9}&9&\mlc{9\\10}&10&\mlc{10\\11}&11&\mlc{11\\12}&12&\mlc{12\\13}&13&\mlc{13\\14}&14&\mlc{14\\15}&15&\\
				\hline
				A1&\mlc{Naročnikove zahteve}&+& & & & & & & & & & & & & & & & & & & & & & & & & & & & & \\
				\hline
				A2&\mlc{Plan projekta}& & & & & & &+& & & & & & & & & & & & & & & & & & & & & & & \\
				\hline
				A3&\mlc{Sistemske specifikacije}& & & & & & & & &+& & & & & & & & & & & & & & & & & & & & & \\
				\hline
				A4&\mlc{Testni vzorci}& & & & & & & & & & & & & & & & & & & & &+& & & & & & & & & \\
				\hline
				A5&\mlc{Testno poročilo}& & & & & & & & & & & & & & & & & & & & & & &+& & & & & & & \\
				\hline
				A6&\mlc{Načrtovalska dokumentacija}& & & & & & & & & & & & & & & & & & & & & & & & & & & & &+& \\
				\hline
				A7&\mlc{Uporabniški priročnik}& & & & & & & & & & & & & & & & & & & & & & & & & & & & &+& \\
				\hline
		\end{tabular}
		}
			\vspace{1cm}
			\begin{itemize}
				\item Legenda:
				\begin{itemize}
					\item planirani čas (+)
					\item dejansko porabljen čas (*)
				\end{itemize}
			\end{itemize}
		\end{center}
		

\end{landscape}
	
\newpage

	\subsection{Pojmovnik}
		
		\begin{center}
				\begin{tabular}{|c|c|}
					\hline
				    \rowcolor{purple!30!}Pojem&Razlaga\\
					\hline
				    \hline
					naročnik&Šaj d.o.o.\\
					\hline
					parametri&\mlc{Količine, s katerimi upravlja program \\(temperatura prostora, relativna vlažnost \\ prostora in osvetljenost prostora)}\\
					\hline
			\end{tabular}
		\end{center}


	\subsection{Priloge}

		\qquad \qquad Opisi uporabljenih strategij

		\subsubsection{Opis strategije: Prisotnost zahtev (Z)}

			\begin{enumerate}
				\item Strategija je uporabna je v vseh primerih, kjer so znane specifikacije 
					  oziroma zahteve, med katerimi ni nobenih relacij. Predpostavka o napaki: določena zahteva 
					  ni implementirana. S to strategijo odkrivamo zahteve, ki niso implementirane. 
					  Razen zelo redkih izjem, ne bomo odkrili napačno implementiranih zahtev in 
					  zahtev, ki so po nepotrebnem implementirane.
    			
			 	\item Testirni model je seznam zahtev.
				\item \textbf{Pravilo za načrtovanje testnih primerov:} Za vsako zahtevo tvori najmanj en testni primer. 
					  Vhodne podatke si poljubno izberi.
				\item Z načrtovanjem testnih primerov lahko začnemo, ko so zahteve postavljene.
				\item Testirna strategija je izčrpana, ko preverimo prisotnost vsake zahteve v seznamu.
			\end{enumerate}

		\subsubsection{Opis strategije za preverjanje robustnosti (R)}
				
			\begin{enumerate}
				\item Strategija je uporabna je v vseh primerih, kjer je zahtevana robustnost in je možno tvoriti opis vhodne domene.
				\item Predpostavka o nepravilnosti: program ni robusten, čeprav bi moral biti. 
					  S to strategijo ne bomo odkrili nepravilnosti, ki se pojavljajo pri procesiranju veljavnih podatkov.
				\item Testirni model je opis vhodne domene.
				\item \textbf{Pravilo za načrtovanje testnih primerov:} V vhodni domeni in identificiraj 
					  prepovedane razrede. Za vsak prepovedan razred tvori en testni primer.
				\item Z načrtovanjem testnih primerov lahko začnemo, ko je opisana vhodna domena.
				\item Testirna strategija je izčrpana, ko smo pokrili vse neveljavne razrede v vhodni 
					  domeni. Zgornje število testnih primerov je enako številu neveljavnih razredov.	
			\end{enumerate}


		\subsubsection{Opis strategije: ugibanje nepravilnosti (U)}
				
			\begin{enumerate}
					\item Strategija je splošno uporabna.
					\item Predpostavlja se, da je prisotna določena nepravilnost ali napaka.
					\item Testirni model je seznam potencialnih nepravilnosti oziroma napak.
					\item \textbf{Pravilo za načrtovanje testnih primerov:} Za vsako potencialno napako 
						  oziroma nepravilnost v seznamu tvorimo en testni primer, s katerim preverimo, 
						  ali je ta napaka/nepravilnost prisotna.
					\item Z načrtovanjem testnih primerov lahko začnemo, ko je imamo pripravljen seznam.
					\item Testirna strategija je izčrpana, ko smo pokrili celoten seznam. Zgornje število 
						  testnih primerov je enako številu napak oziroma nepravilnosti v seznamu.	
			\end{enumerate}

		\subsubsection{Opis strategije: Mejne vrednosti (M)}
				
			\begin{enumerate}
			  		\item Strategija je splošno uporabna.
					\item Predpostavka o nepravilnosti: vhodni podatki, ki se nahajajo v okolici 
						  ali pa točno na meji med veljavnim in neveljavnim območjem, se bodo nepravilno procesirali.
					\item Testirni model je vhodna in izhodna domena.
					\item \textbf{Pravilo za načrtovanje testnih primerov:} določi meje med veljavnimi in neveljavnimi 
						  podatki. Izberi vrednost točno na meji, malo nad in malo pod njo.
					\item Z načrtovanjem testnih primerov lahko začnemo, ko je imamo podatkovni slovar.
					\item Testirna strategija je izčrpana, ko smo uporabili vse meje.	
			\end{enumerate}
\end{document}
